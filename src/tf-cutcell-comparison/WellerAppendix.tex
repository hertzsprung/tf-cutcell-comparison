\section{Semi-implicit treatment of the Hodge operator\label{appx:SI}}

In order to ensure curl-free pressure gradients, following \citet{weller-shahrokhi2014},
the co-variant momentum component, that is the momentum at the cell
face in the direction between cell centres is used as the prognostic
variable for velocity:
\begin{equation}
V_{f}=\rho_{f}\mathbf{u}_{f}\cdot\mathbf{d}_{f}
\end{equation}
where $\mathbf{d}_{f}$ is the vector between cell centres and subscript
$f$ means ``at face $f$''. Fluxes across faces, or the contravariant
component, are diagnostic variables:
\begin{equation}
U_{f}=\rho_{f}\mathbf{u}_{f}\cdot\mathbf{S}_{f}
\end{equation}
where $\mathbf{S}_{f}$ is the outward pointing normal vector to face
$f$ with magnitude equal to the area of the face. $U$ is the vector
of all values of $U_{f}$ and $V$ is the vector of all values of
$V_{f}$ then we can define the Hodge operator: a matrix to transform
$V$ to $U$:
\begin{equation}
U=HV.
\end{equation}
For energy conservation, \citet{TC12} showed that the Hodge operator
must be symmetric and positive definite. We define a symmetric $H$
suitable for arbitrary 3D meshes:
\begin{equation}
U_{f}=(\rho\mathbf{u})_{F}\cdot\mathbf{S}_{f}
\end{equation}
where subscript $F$ implies mid-point interpolation from two surrounding
cell values onto face $f$:
\begin{equation}
(\rho\mathbf{u})_{F}=\frac{1}{2}\sum_{c\in f}(\rho\mathbf{u})_{C}
\end{equation}
where $c\in f$ means the two cells either side of cell $c$. $(\rho\mathbf{u})_{C}$
is the consistent cell centre reconstruction of $\rho\mathbf{u}$
from surrounding values of $V_{f}$:
\[
(\rho\mathbf{u})_{C}=\left(\sum_{f^{\prime}\in c}\mathbf{d}_{f^{\prime}}\otimes\mathbf{d}_{f^{\prime}}^{T}\right)^{-1}\sum_{f^{\prime}\in c}\mathbf{d}_{f^{\prime}}V_{f^{\prime}}
\]
where $\mathbf{d}_{f^{\prime}}\otimes\mathbf{d}_{f^{\prime}}^{T}$
is a $3\times3$ tensor and so the inversion of the tensor sum is
a local operation which can be calculated once for each cell of the
grid before time-stepping begins. The $H$ implied by this reconstruction
of $U$ is likely to be positive definite for meshes with sufficiently
low non-orthogonality. However this has not been proved. 

The semi-implicit technique involves combining the momentum (\ref{eq:momentum}),
continuity (\ref{eq:cont}) and $\theta$ (\ref{eq:theta}) equations
and the equation of state (\ref{eq:state}) to form a Helmholtz equation
to be solved implicitly (see \citet{weller-shahrokhi2014}). The semi-implicit
solution technique with a Hodge operator can be defined by considering
only a discretised from of the continuity equation:
\begin{equation}
\frac{\phi^{(n+1)}-\rho^{(n)}}{\Delta t}+\frac{1}{2}\left\{ \nabla\cdot(HV)^{(n)}+\nabla\cdot(HV)^{(n+1)}\right\} =0.\label{eq:contD}
\end{equation}
The divergence is discretised using Gauss' divergence theorem so that:
\begin{equation}
\nabla\cdot\left(HV\right)=\frac{1}{V_{c}}\sum_{f\in c}n_{f}(HV)_{f}\label{eq:GaussDiv}
\end{equation}
where $V_{c}$ is the volume of cell $c$ and $n_{f}=1$ if $\mathbf{d}_{f}$
points outwards from the cell and $n_{f}=$-1 otherwise. Equation
(\ref{eq:GaussDiv}) is now a sum over a sum since $(HV)_{f}$ is
one element of a matrix-vector multiply. In order to simplify the
construction of the matrix for the Helmholtz problem, only the diagonal
terms of $HV$ are treated implicitly. We therefore separate $H$
into a diagonal and off-diagonal matrix:
\begin{equation}
H=H_{d}+H_{off}.
\end{equation}
Eqn (\ref{eq:contD}) can now be approximated by:
\begin{equation}
\frac{\phi^{(n+1)}-\rho^{(n)}}{\Delta t}+\frac{1}{2}\left\{ \nabla\cdot(HV)^{(n)}+\nabla\cdot(H_{d}V)^{(n+1)}+\nabla\cdot(H_{off}V)^{\ell}\right\} =0\label{eq:contDd}
\end{equation}
where superscript $\ell$ means lagged values taken from a previous
iteration or from a previous stage of a Runge-Kutta scheme. This was
the approach taken by \citet{weller-shahrokhi2014}. However, numerical
solution of eqn (\ref{eq:contDd}) turns out to be unstable when using
a large time-step on a highly non-orthogonal mesh associated with
terrain following coordinates over steep orography. Improved stability
and energy conservation can be achieved by splitting $H$ into a diagonal
component which would be correct on an orthogonal grid and a non-orthogonal
correction:
\begin{equation}
H=H_{c}+H_{corr}
\end{equation}
where the diagonal matrix $H_{c}=|\mathbf{S}_{f}|/|\mathbf{d}_{f}|$
and the non-orthogonal correction is $H_{corr}=H-H_{c}$. The orthogonal
part, $H_{c}$, can be treated implicitly in the Helmholtz equation:
\begin{equation}
\frac{\phi^{(n+1)}-\rho^{(n)}}{\Delta t}+\frac{1}{2}\left\{ \nabla\cdot(HV)^{(n)}+\nabla\cdot(H_{c}V)^{(n+1)}+\nabla\cdot(H_{corr}V)^{\ell}\right\} =0.\label{eq:contDc}
\end{equation}
This form is used for the solutions of the Euler equations in this
paper and is stable, with good energy conservation for all of the
tests presented. 

