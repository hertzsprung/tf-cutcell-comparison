\section{Semi-implicit treatment of the Hodge operator\label{appx:SI}}

In order to ensure curl-free pressure gradients, following \citet{WS14},
the momentum at the cell face in the direction between cell centres
is used as the prognostic variable for velocity:
\begin{equation}
V_{f}=\rho_{f}\mathbf{u}_{f}\cdot\mathbf{d}_{f}
\end{equation}
where $\mathbf{d}_{f}$ is the vector between cell centres and subscript
$f$ means ``at face $f$''. Fluxes across faces are diagnostic
variables:
\begin{equation}
U_{f}=\rho_{f}\mathbf{u}_{f}\cdot\mathbf{S}_{f}
\end{equation}
where $\mathbf{S}_{f}$ is the outward pointing normal vector to face
$f$ with magnitude equal to the area of the face. $U$ is the vector
of all values of $U_{f}$ and $V$ is the vector of all values of
$V_{f}$ then we can define the Hodge operator: a matrix to transform
$V$ to $U$:
\begin{equation}
U=HV.
\end{equation}
For energy conservation, \citet{TC12} showed that the Hodge operator
must be symmetric and positive definite. We define a symmetric $H$
suitable for arbitrary 3D meshes:
\begin{equation}
U_{f}=(\rho\mathbf{u})_{F}\cdot\mathbf{S}_{f}
\end{equation}
where subscript $F$ implies mid-point interpolation from two surrounding
cell values onto face $f$:
\begin{equation}
(\rho\mathbf{u})_{F}=\frac{1}{2}\sum_{c\in f}(\rho\mathbf{u})_{C}
\end{equation}
where $c\in f$ means the two cells either side of cell $c$ and where
$(\rho\mathbf{u})_{C}$ is the consistent cell centre reconstruction
of $\rho\mathbf{u}$ from surrounding values of $V_{f}$:
\[
(\rho\mathbf{u})_{C}=\left(\sum_{f^{\prime}\in c}\mathbf{d}_{f^{\prime}}\otimes\mathbf{d}_{f^{\prime}}^{T}\right)^{-1}\sum_{f^{\prime}\in c}\mathbf{d}_{f^{\prime}}V_{f^{\prime}}
\]
where $\mathbf{d}_{f^{\prime}}\otimes\mathbf{d}_{f^{\prime}}^{T}$
is a $3\times3$ tensor and so the inversion of the tensor sum is
a local operation which can be calculated once for each cell of the
grid before time-stepping begins. 

