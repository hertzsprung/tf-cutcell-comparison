\documentclass[twocol]{ametsoc}
\journal{mwr}

\usepackage{xcolor}
\usepackage{amsmath}
\usepackage{mathtools}
\usepackage{siunitx}

\bibpunct{(}{)}{;}{a}{}{,}

\title{Comparison of Terrain Following and Cut Cell Grids using a Non-Hydrostatic Model}
\authors{James Shaw\correspondingauthor{Department of Meteorology, University of Reading, Earley Gate, PO Box 243, Reading, RG6 6BB, UK.} and Hilary Weller}
\affiliation{Department of Meteorology, University of Reading, Reading, United Kingdom}
\email{js102@zepler.net}

\abstract{Terrain following coordinates are widely used in operational models but the cut cell method has been proposed as an alternative that can more accurately represent the orography and associated atmospheric dynamics.  Because the type of grid is usually chosen during model implementation, it is typically necessary to use different models in order to compare the accuracy of different grids.  On the contrary, we use a single C-grid finite volume model to enable a like-for-like comparison of terrain following and cut cell grids.
A series of standard two-dimensional tests using idealised terrain are performed: tracer advection in a prescribed horizontal velocity field, a test of a stably stratified atmosphere at rest, and orographically induced gravity waves.  In addition, we formulate a new tracer advection test having a velocity field that is everywhere tangential to the terrain following coordinate surfaces.  This new test presents a challenge to cut cells in situations with a large Froude number and large Scorer parameter.
The results of the advection tests demonstrate that tracer accuracy depends upon alignment of the flow with the grid.  As expected, the cut cell grid maintains greater accuracy in the test of stationary, stratified flow.  In the gravity waves test, results on all grids are in good agreement with existing results from the literature.  However, we find that the Lorenz computational mode is excited as a vertical zig--zag in potential temperature on the cut cell grid.}

\begin{document}

\maketitle

\section{Introduction}
Representing orography accurately in numerical weather prediction systems is necessary to model downslope winds and local precipitation.  Orography also exerts strong non-local influences from the heat release due to convection and by the atmospheric response to form drag and gravity wave drag.  There are two main approaches to represent orography on a grid: terrain following layers and cut cells.  Both methods align cells in vertical columns.  Because most models are designed for a particular type of grid, existing studies of cut cell solutions have compared results with terrain following grid solutions implemented within different models, for example \citet{good2014}.  Instead, this study uses a single model to enable a like-for-like comparison between solutions using terrain following and cut cell grids.

With increasing horizontal model resolution, the discrete representation of terrain can become steeper, making accurate calculation of the horizontal pressure gradient more difficult when using terrain following layers \citep{gary1973,steppeler2002}.  Numerical errors in this calculation result in spurious winds and can cause numerical instability \citep{fast2003,webster2003}.  Cut cell methods seek to reduce the error that is associated with steep orography.

With terrain following (TF) layers the terrain's influence decays with height so that the bottommost layers follow the underlying surface closely while the uppermost layers are flat.  There are two main approaches to minimizing errors associated with TF layers.  First, by smoothing the effects of terrain with height, the influence of the terrain is reduced, hence errors in the calculated horizontal pressure gradient are also reduced aloft \citep{schaer2002,leuenberger2010,klemp2011}.  However, the error is not reduced at the ground where steep terrain remains unmodified.

Second, numerical errors can also be reduced by improving the accuracy in calculating the horizontal pressure gradient itself.  TF layers are usually implemented using a coordinate transformation onto a rectangular computational domain, which introduces metric terms into the equations of motion.  The techniques proposed by \citet{klemp2011} and \cite{zaengl2012} both involve the calculation of the horizontal pressure gradient in the physical domain.  This gave them the flexibility to design more accurate horizontal pressure gradient discretizations using more appropriate stencils.

Despite their associated numerical errors, TF layers are in widespread operational use \citep{steppeler2003}.  They are attractive because their rectangular structure is simple to process by computer, boundary layer resolution can be increased with variable spacing of vertical layers \citep{schaer2002}, and cell sizes remain almost constant \citep{jebens2011}.

Cut cells is an alternative method in which the grid does not follow the terrain but, instead, cells that lie entirely below the terrain are removed, and those that intersect the surface are modified in shape so that they more closely fit the terrain.  The resulting grid is orthogonal everywhere except near cells that have been cut.  Hence, errors are still introduced when calculating the horizontal pressure gradient between cut and uncut cells.

The cut cell method can create some very small cells which reduce computational efficiency \citep{klein2009}, and several approaches have been tried to alleviate the problem \citep{steppeler2002,yamazaki-satomura2010,jebens2011}.

Several studies have shown examples where cut cells produce more accurate results when compared to TF coordinates.  Spurious winds seen in TF coordinates are not present and errors do not increase with steeper terrain \citep{good2014}.  A comparison of TF and cut cells using real initial data by \citet{steppeler2013} found that precipitation and wind forecasts were more accurately in the cut cell model.

This study uses the nonhydrostatic model from \citet{weller-shahrokhi2014} to enable a like-for-like comparison between terrain following and cut cell grids for idealised, two-dimensional test cases from the literature.  Section 2 presents the formulation of the terrain following and cut cell grids used in the experiments that follow.  In section 3 we outline the model from \citet{weller-shahrokhi2014} and the governing equations.  Section 4 analyses the results from two tracer advection tests, a test of a stably stratified atmosphere initially at rest, and orographically induced gravity waves.  Concluding remarks are made in section 5.

\begin{figure*}
	\centering
%	\includegraphics{../fig-meshes/fig-meshes.pdf}
	%
	\caption{Examples of (a) BTF, (b) SLEVE, and (c) a cut cell grid, showing cell edges in the lowest four layers.  The two dimensional grids are \SI{20}{\kilo\meter} wide and \SI{20}{\kilo\meter} high.  SLEVE parameters are specified in the resting atmosphere test in section~\ref{sec:results}\ref{sec:resting}.  The cut cell grid was created by intersecting the terrain surface with a regular grid as described in section~\ref{sec:grid}.}
	\label{fig:grid}
\end{figure*}

\section{Grids}
\label{sec:grid}

Here we describe the formulation of the terrain following grids and the method of cut cell grid construction.  The techniques presented are used to define the grids for the experiments in the subsequent section.

\citet{galchen-somerville1975} proposed a basic terrain following (BTF) coordinate defined as 
\begin{equation}
	z = \left( H - h \right) \left( z^\star / H \right) + h \label{eqn:btf}
\end{equation}
where, in two dimensions, \(z(x, z^\star)\) is the physical height of the coordinate surface at level \(z^\star\), \(H\) is the height of the domain, and \(h(x)\) is the height of the terrain surface.  This formulation results in $z^\star/H$ ranging from 0 to 1.  Using this coordinate, the terrain's influence decays linearly with height but disappears only at the top of the domain.  An example is shown in figure~\ref{fig:grid}a.

The sigma coordinate transform of \citet{phillips1957} is equivalent to the BTF coordinate transform since they both decay linearly.  However, because \(\sigma = p/p_s\) decays with pressure rather than height, sigma coordinates also vary with surface pressure, \(p_s(x, t)\).

The hybrid terrain following (HTF) coordinates of \citet{simmons-burridge1981} improve upon BTF coordinates by using a decay function that allows the influence of the terrain to be removed at a specified height, producing flat model layers aloft.

The smooth level vertical (SLEVE) coordinate proposed by \citet{schaer2002} achieves a more regular TF grid in the middle and top of the domain than the BTF coordinate.  The terrain height is split into large-scale and small-scale components, \(h_1\) and \(h_2\), such that \(h = h_1 + h_2\), with each component having a different exponential decay. The transformation is defined as 
\begin{align}
	z &= z^\star + h_1 b_1 + h_2 b_2
\intertext{where the vertical decay functions are given by}
	b_i &= \frac{\sinh \left( \left( H / s_i \right)^n - \left( z^\star / s_i \right)^n \right)}{\sinh \left( H / s_i \right)^n}
\end{align}
with \(s_1\) and \(s_2\) are the scale heights of large-scale and small-scale terrain respectively.  The exponent \(n\) was introduced by \citet{leuenberger2010} in order to increase cell thickness in the layers nearest the ground, allowing longer timesteps.  \citet{leuenberger2010} found the exponent has an optimal value of \(n = 1.35\).  Choosing \(n = 1\) gives the decay functions used by \citet{schaer2002}.  An example of the SLEVE grid can be seen in figure~\ref{fig:grid}b.

Most implementations of terrain following layers use a coordinate system that makes the domain rectangular, but introduces metric terms into the equations of motion.  Instead, the model employed in this study uses Cartesian coordinates and unstructured grids.  By doing so, results from the same model can be compared between terrain following and cut cell grids without modifying the equation set or discretisation.

The OpenFOAM utility \texttt{snappyHexMesh} was used to create a grid that approximates the cut cell method.  First, a custom utility was used to move points beneath the surface up to the surface creating small cells near mountain peaks.  Second, the surface faces were taken from the BTF grid and \texttt{snappyHexMesh} was used to intersect the surface with the grid.  This tool removes cells whose centres are below the surface and displaces boundary vertices so that they are `snapped' to the BTF surface \citep{openfoam2015}.  An example of the resulting grid is shown in figure~\ref{fig:grid}c.

There are two details of grid construction which mean that resulting cut cell grids can differ slightly from a typical cut cell grid created using a shaving method, as described by \citet{adcroft1997}.  First, when \texttt{snappyHexMesh} moves vertices onto the terrain surface, some points are moved horizontally.  Second, the utility does not create new points necessary for pentagonal cells.


\section{Model}
\label{sec:model}

We use the finite volume model from \citet{weller-shahrokhi2014} which details a C-grid discretisation of the fully-compressible Euler equations, given by
\begin{subequations}
\begin{align}
	\text{Momentum} &\ &\  	\frac{\partial \rho \bm{u}}{\partial t} + \nabla \cdot \rho \bm{uu} &= \rho \bm{g} - c_p \rho \theta \nabla \Pi \label{eq:momentum} \\
	\text{Continuity} &\ &\	\frac{\partial \rho}{\partial t} + \nabla \cdot \rho \bm{u} &= 0 \\
	\text{\(\rho \theta\) advection} &\ &\ \frac{\partial \rho \theta}{\partial t} + \nabla \cdot \rho \bm{u} \theta &= 0 \\
	\text{Equation of state} &\ &\ \Pi^{(1 - \kappa)/\kappa} &= \frac{R \rho \theta}{p_0}
\end{align}
\end{subequations}
where \(\rho\) is the density, \(\bm{u}\) is the velocity field, \(\bm{g}\) is the gravitational acceleration, \(c_p\) is the heat capacity at constant pressure, \(\theta = T \left(p_0/p\right)^\kappa\) is the potential temperature, \(T\) is the tempereature, \(p\) is the pressure, \(p_0\) is a reference pressure, \(\Pi = \left(p / p_0 \right)^\kappa\) is the Exner function of pressure, and \(\kappa = R/c_p\) is the gas constant to heat capacity ratio.

The model uses a Lorenz C-grid staggering of thermodynamic variables such that $\theta$, $\rho$ and $\Pi$ are stored at cell centres and velocities at cell faces.

A minor change has been made to the curl-free, non-orthogonal formulation as described in \citet{weller-shahrokhi2014} to improve stability on non-orthogonal grids.  \citet{weller-shahrokhi2014} use a Hodge dual operator, \(H\), to map from co-variant to contra-variant velocity components:
\begin{align}
	U = HV
\end{align}
where \(U = \rho\bm{u}\cdot\bm{S}_f\) and \(V=\rho\bm{u}\cdot\bm{d}_f\).  \(\bm{S}_f\) is the surface normal vector for a face, \(f\), which has a magnitude of the surface area, and \(\bm{d}_f\) is the vector between neighbouring cell centres.

In order to create a semi-implicit formulation, \(H\) is split into diagonal and off-diagonal components and the off-diagonal components are treated explicitly in the Helmholtz equation.  However, this was found to be unstable when using a large timestep on a grid with large non-orthogonality.  Improved stability is achieved by splitting \(H\) into the orthogonal part and the non-orthogonal correction: 
\begin{align}
	H = H_c + H_\mathrm{corr}
\end{align}
where the diagonal matrix \(H_c=\frac{|\bm{S}_f|}{|\bm{d}_f|}\) and the non-orthogonal correction is simply \(H_\mathrm{corr} = H - H_c\).  The orthogonal part, \(H_c\), is treated implicitly in the Helmholtz equation and the non-orthogonal part, \(H_\mathrm{corr}\), is treated explicitly. 

For tracer advection tests, the advection equation in flux form, given by
\begin{align}
\partial \phi / \partial t + \nabla \cdot \left( \bm{u} \phi \right) = 0
\end{align}
is solved using the upwind-biased multidimensional cubic scheme from \citet{weller-shahrokhi2014} which is non-monotonic and not flux corrected.
The time derivative is solved using a three-stage, second order Runge-Kutta scheme:
\begin{subequations}
\begin{align}
	\varphi^\star &= \varphi^{(n)} + \Delta t f(\phi^{(n)}) \\
	\varphi^{\star\star} &= \varphi^{(n)} + \frac{\Delta t}{2} \left( f(\phi^{(n)}) + f(\phi^\star) \right) \\
	\varphi^{(n+1)} &= \varphi^{(n)} + \frac{\Delta t}{2} \left( f(\phi^{(n)}) + f(\phi^{\star\star}) \right)
\end{align}
\end{subequations}
where \(f(\phi^{(n)}) = - \nabla \cdot (\bm{u} \varphi^{(n)})\) at time level \(n\).

\section{Results}
\label{sec:results}

A series of two-dimensional tests are performed over idealised orography.  For each test, results on the BTF, SLEVE and cut cell grid are compared.  The first test from \citet{schaer2002} advects a tracer in a horizontal velocity field.  Second, a new tracer advection test is formulated employing a terrain following velocity field to challenge the advection scheme on cut cell grids.  The third test solves the Euler equations for a stably stratified atmosphere initially at rest, following \citet{klemp2011}.  Finally, as specified by \citet{schaer2002}, a test of orographically-induced gravity waves is performed.  No explicit diffusion is used in any of the tests.

The OpenFOAM implementation of the numerical model and grid generation utilities are available at \url{https://github.com/hertzsprung/AtmosFOAM/tree/shaw-weller-2015-mwr}.  The test cases are located at \url{https://github.com/hertzsprung/AtmosTests/tree/shaw-weller-2015-mwr}.


\subsection{Horizontal advection}

Following \citet{schaer2002}, a tracer is transported above wave-shaped terrain by solving the advection equation for a prescribed horizontal wind.  This test challenges the accuracy of the advection scheme in the presence of grid distortions.

The domain is \SI{301}{\kilo\meter} wide and \SI{25}{\kilo\meter} high, discretized onto a grid with \(\Delta x = \SI{1}{\kilo\meter}\) and \(\Delta z^\star = \SI{500}{\meter}\).  The domain specified by \citet{schaer2002} is \SI{300}{\kilo\meter} between the outermost cell centres where tracer values are specified.  In order to reproduce the result from \citet{schaer2002}, the domain has been extended horizontally by \(\Delta x/2\) \si{\meter} in both directions so that the distance between the outermost cell centres is still \SI{300}{\kilo\meter}.

The terrain is wave-shaped, specified by the surface height, \(h\), such that
\begin{subequations}
\begin{align}
   h(x) &= h^\star \cos^2 ( \alpha x )
%
\intertext{where}
%
   h^\star(x) &= \left\{ \begin{array}{l l}
       h_0 \cos^2 ( \beta x ) & \quad \text{if $| x | < a$} \\
	0 & \quad \text{otherwise}
    \end{array} \right.
\end{align}
\end{subequations}
where $a = \SI{25}{\kilo\meter}$ is the mountain envelope half-width, $h_0 = \SI{3}{\kilo\meter}$ is the maximum mountain height, $\lambda = \SI{8}{\kilo\meter}$ is the wavelength, \(\alpha = \pi / \lambda\) and \(\beta = \pi / 2a\).  On the SLEVE grid, the large-scale component $h_1$ is given by \(h_1(x) = h^\star(x) / 2\)
and $s_1 = \SI{15}{\kilo\meter}$ is the large scale height, and $s_2 = \SI{2.5}{\kilo\meter}$ is the small scale height.  The optimisation of SLEVE by \citet{leuenberger2010} is not used, so the exponent $n = 1$.

The wind is entirely horizontal and is prescribed as
\begin{align}
	u(z) = u_0 \left\{ \begin{array}{l l}
		1 & \quad \text{if $z \geq z_2$} \\
		\sin^2 \left( \frac{\pi}{2} \frac{z - z_1}{z_2 - z_1} \right) & \quad \text{if $z_1 < z < z_2$} \\
		0 & \quad \text{otherwise}
	\end{array} \right.	
\end{align}
where $u_0 = \SI{10}{\meter\per\second}$, $z_1 = \SI{4}{\kilo\meter}$ and $z_2 = \SI{5}{\kilo\meter}$.
This results in a constant wind aloft, and zero flow at \SI{4}{\kilo\meter} and below.

To ensure that the discrete velocity field is non-divergent, velocities are prescribed at cell faces by differencing the streamfunction, \(\Psi\), along the edges from \(\Psi\) stored at cell vertices.  Since \(u = - \partial \Psi / \partial z\), the streamfunction is found by vertical integration of the velocity profile:
\begin{align}
	\Psi(z) &= \frac{u_0}{2} \left\{ \begin{array}{l l}
		\left( 2z - z_1 - z_2 \right) & \enskip \text{if $z > z_2$} \\
		z - z_1 - \frac{z_2 - z_1}{\pi} \sin \left(\pi \frac{z - z_1}{z_2-z_1}\right) & \enskip \text{if $z_1 < z \leq z_2$} \\
		0 & \enskip \text{if $z \leq z_1$}
	\end{array} \right.
\end{align}

A tracer, $\varphi$, is positioned upstream above the height of the terrain.  It has the shape
\begin{align}
	\varphi(x, z) &= \varphi_0 \left\{ \begin{array}{l l}
		\cos^2 \left( \frac{\pi r}{2} \right) & \quad \text{if $r \leq 1$} \\
		0 & \quad \text{otherwise}
	\end{array} \right.
%
\intertext{having radius, $r$, given by}
%
	r &= \sqrt{
		\left( \frac{x - x_0}{A_x} \right)^2 + 
		\left( \frac{z - z_0}{A_z} \right)^2
	}
\end{align}
where $A_x = \SI{25}{\kilo\meter}$, $A_z = \SI{3}{\kilo\meter}$ are the horizontal and vertical half-widths respectively, and $\varphi_0 = 1$ is the maximum magnitude of the anomaly.  At $t = \SI{0}{\second}$, the anomaly is centred at $(x_0, z_0) = (\SI{-50}{\kilo\meter}, \SI{9}{\kilo\meter})$ so that the anomaly is upwind of the mountain and well above the maximum terrain height of \SI{3}{\kilo\meter}.  Analytic solutions can be found by setting the anomaly centre such that $x_0 = ut$.

Unlike \citet{schaer2002} who use periodic lateral boundaries, a fixed value of 0 is used at the inlet boundary and all other boundaries have zero gradient.  Furthermore, a second order Runge-Kutta timestepping scheme is used here instead of the leapfrog scheme used by \citet{schaer2002}.
Tests are integrated forward in time for \SI{10000}{\second} with a timestep of \(\Delta t = \SI{25}{\second}\).

The test was executed on the BTF, SLEVE and cut cell grids, and on a regular grid with flat terrain using a centred linear scheme and the upwind-biased cubic scheme.   Results were also obtained on BTF and SLEVE grids with the fourth order scheme from \citet{schaer2002} using a modified version of their Fortran code.  The original version of the code interpolated the geometric height at cell vertices from values at adjacent half levels in order to calculate the streamfunction, \(\Psi(z)\).  The modified version used here directly calculates the height at vertices to enable comparisons with the finite volume model solutions.

Tracer contours at \(t = \SI{0}{\second}, \SI{5000}{\second}\) and \(\SI{10000}{\second}\) are shown in Figure~\ref{fig:advection-tracer}.
The results are compared on the BTF grid for the centred linear scheme (\ref{fig:advection-tracer}a) and the fourth order scheme from \citet{schaer2002} (\ref{fig:advection-tracer}c), and the upwind-biased cubic scheme on the cut cell grid (\ref{fig:advection-tracer}b), and BTF grid (\ref{fig:advection-tracer}d).  Tracer errors at \(t = \SI{10000}{\second}\) are shown for the fourth order and upwind-biased cubic schemes in figures \ref{fig:advection-tracer}e and \ref{fig:advection-tracer}f respectively.

By \(t = \SI{10000}{\second}\), the tracer suffers from distortion on the BTF grid using the centred linear scheme and some artefacts remain about the mountain peak.  The tracer has spread vertically due to increased numerical errors when the tracer is transported between layers.  Distortions are reduced by using the fourth order scheme from \citet{schaer2002} (figure~\ref{fig:advection-tracer}c), but the computational mode is seen as a grid-scale oscillation that travels in the opposite direction to the wind (figure~\ref{fig:advection-tracer}e).  The results from the centred linear and fourth order schemes are slightly worse than the respective results from \citet{schaer2002} (their figure 6a and figure 8) because of the difference in the discretisation of $\Psi$.

Using the upwind-biased cubic scheme, tracer magnitude and shape are well-preserved on all grids, both above the mountain at \(t = \SI{5000}{\second}\) and past the mountain at \(t = \SI{10000}{\second}\).  In this test, advection is most accurate on the cut cell grid (figure~\ref{fig:advection-tracer}b) and regular grid (not shown).  As found by \citet{good2014}, the result is the same on both grids.  This is to be expected since the wind is zero in the region of the ground and flow aloft is aligned with the grids.  On the BTF grid, the tracer is less distorted by the cubic upwind-biased scheme (figure~\ref{fig:advection-tracer}d) compared to the centred linear scheme (figure~\ref{fig:advection-tracer}a) or fourth order scheme (figure~\ref{fig:advection-tracer}c).

Minimum and maximum tracer values and \(\ell^2\) error norms on the BTF, SLEVE, cut cell and regular grids are summarised in table~\ref{tab:advection}, where the \(\ell^2\) error norm is defined as 
\begin{align}
	\ell^2 &= \sqrt{\frac{\sum_c \left( \varphi - \varphi_{T} \right)^2 \mathcal{V}_c}{\sum_c \left( \varphi_T^2 \mathcal{V}_c \right)}}
\end{align}
where $\varphi$ is the numerical tracer value, $\varphi_T$ is the analytic value and $\mathcal{V}_c$ is the cell volume.

The results of the cubic upwind-biased scheme on TF and regular grids are comparable with those for the fourth-order centred scheme from \citet{schaer2002}.  Error is largest on the BTF grid with \(\ell^2 = \num{0.00791}\) but significant reduced on the SLEVE grid with \(\ell^2 = \num{0.00108}\).  The error is approximately halved by changing from the SLEVE grid to the cut cell grid.
Tracer minima and maxima for the centred linear and fourth order schemes are lower than those presented by \citet{schaer2002} because no interpolation is used to calculate the streamfunction.

The centred linear scheme is sensitive to changes in the terrain profile: when the domain width is reduced to \SI{300}{\kilo\meter}, the $\ell^2$ error on the BTF grid rises from \num{0.0210} to \num{0.0318}.  The upwind-biased cubic scheme is less sensitive to the same change, with the $\ell^2$ error decreasing from \num{0.00791} to \num{0.00767}.

The results of the horizontal advection test show that numerical errors are due either to misalignment of the flow with the grid, or to grid distortions.  Using the upwind-biased cubic scheme, distortions in the grid do not significantly distort the tracer.

\begin{figure*}
	\centering
	\includegraphics{../fig-advection-tracer/fig-advection-tracer.pdf}
%
	\caption{Horizontally advected tracer contours at \(t = \SI{0}{\second}\), \SI{5000}{\second} and \SI{10000}{\second} using (a) centred linear scheme on a BTF grid, the upwind-biased cubic scheme on (b) the cut cell grid, (d) the BTF grid, and (c) the fourth order scheme from \citet{schaer2002} on the BTF grid, with contour intervals every 0.1.  Errors on the BTF grid at \(t = \SI{10000}{\second}\) are shown for (e) the fourth order scheme from \citet{schaer2002}, and (f) the upwind-biased cubic scheme, with contour intervals every 0.01.  Negative contours denoted by dotted lines.  The terrain profile is also shown immediately above the $x$ axis.  Subfigures (c) and (e) produced using the modified version of the code from \citet{schaer2002}.}
	\label{fig:advection-tracer}
\end{figure*}


\subsection{Terrain following advection}
In the horizontal advection test, results were more accurate where the flow was aligned with the grid layers, and distortions in the BTF grid led to increased errors.  To determine the primary source of numerical error, we formulate a new tracer advection test in which the velocity field is everywhere tangential to the basic terrain following coordinate surfaces.  In this new test, flow is aligned with the BTF grid layers.  Misalignment on the SLEVE and cut cell grids is designed to challenge the advection scheme.

The spatial domain, mountain profile, initial tracer profile and discretisation are the same as those in the horizontal tracer advection test.  The velocity field is defined using a streamfunction, $\Psi$, so that the continuous velocity field is non-divergent and follows the BTF coordinate surfaces given by equation~\ref{eqn:btf} such that
\begin{equation}
	\Psi(x,z) = - u_0 H \frac{z - h}{H - h}
\end{equation}
where $u_0 = \SI{10}{\meter\per\second}$, which is the horizontal wind speed where $h(x) = 0$.
The horizontal and vertical components of velocity, $u$ and $w$, are then given by
\begin{align}
	u &= - \frac{\partial \Psi}{\partial z} = - u_0 \frac{H}{H - h}, \quad w = \frac{\partial \Psi}{\partial x} = u_0 H \frac{\mathrm{d} h}{\mathrm{d} x} \frac{z - H}{\left( H - h \right)^2}, \nonumber \\
	\frac{\partial h}{\partial x} &= - h_0 \left[ 
		\beta \cos^2 \left( \alpha x \right) \sin \left( 2 \beta x \right) +
		\alpha \cos^2 \left( \beta x \right) \sin \left( 2 \alpha x \right)
	\right]
\end{align}
Unlike the horizontal advection test, flow extends from the top of the domain all the way to the ground.  The discrete velocity field is calculated using the streamfunction in the same way as the horizontal advection test.

At $t = \SI{10000}{\second}$ the tracer has passed over the mountain.  The horizontal position of the tracer centre can be calculated by integrating along the trajectory to find $t$, the time taken to pass from one side of the mountain to the other:
\begin{align}
	\mathrm{d}t &= \mathrm{d}x / u(x) \\
	t &= \int_0^x \frac{H - h(x)}{u_0 H}\:\mathrm{d}x \\
	t &= \frac{x}{u_0} - \frac{h_0}{16 u_0 H} \left[ 4x + \frac{\sin 2 (\alpha + \beta) x}{\alpha + \beta} \right.+ \nonumber \\
   &\ \left. \frac{\sin 2(\alpha - \beta) x}{\alpha - \beta} + 2 \left( \frac{\sin 2\alpha x}{\alpha} + \frac{\sin 2\beta x}{\beta} \right) \right]
\end{align}
Hence, we find that \(x(t=\SI{10000}{\second}) = \SI{51577.4}{\meter}\).  Because the velocity field is non-divergent, the flow accelerates over mountain ridges and the tracer travels \SI{1577.4}{\meter} further compared to advection in the purely horizontal velocity field.  Tracer height is unchanged downwind of the mountains because advection is along the terrain following coordinate surface.

\begin{table*}[t]
	\caption{Minimum and maximum tracer magnitudes and \(\ell^2\) error norms at \(t = \SI{10000}{\second}\) in the horizontal and terrain following tracer advection tests using centred linear and cubic upwind-biased schemes.  For the horizontal advection test, minimum and maximum values are given for the fourth order scheme using the modified code from \citet{schaer2002}.}
\label{tab:advection}
%
\centering
\footnotesize
\begin{tabular}{l l l l S S S S}
\hline\hline
                  &                     &                  & Analytic & {BTF}   & {SLEVE}         & {Cut cell}     & {No terrain} \\
\hline
Horizontal        & Centred linear      & \(\ell^2\) error & 0        & 0.0210  & 0.00233         & 0.00224        & 0.00223     \\
                  &                     & min              & 0        & -0.275  & -0.0252         & -0.0251        & -0.0251     \\
                  &                     & max              & 1        & 0.925   & 0.985           & 0.985          & 0.985       \\
                  & Fourth order        & min              & 0        & -0.0926 & -0.00174        & {---}          & -0.00178    \\
                  &                     & max              & 1        & 1.00    & 0.984           & {---}          & 0.983       \\
                  & Cubic upwind-biased & \(\ell^2\) error & 0        & 0.00791 & 0.00108         & 0.000577       & 0.000576    \\
                  &                     & min              & 0        & -0.0446 & -0.0106         & -0.000674      & -0.00674    \\
                  &                     & max              & 1        & 0.925   & 0.982           & 0.983          & 0.983       \\
\hline
Terrain following & Centred linear      & \(\ell^2\) error & 0        & 0.00251 & 0.0173          & 13.6           & {---}        \\
		  &                     & min              & 0        & -0.0245 & -0.120          & -623           & {---}        \\
                  &                     & max              & 1        & 0.985   & 0.950           & 3480           & {---}        \\
		  & Cubic upwind-biased & \(\ell^2\) error & 0        & 0.00154 & 0.0120          & 0.0134         & {---}        \\
                  &                     & min              & 0        & -0.0110 & -0.0263         & -0.028         & {---}        \\
                  &                     & max              & 1        & 0.983   & 0.865           & 0.851          & {---}        \\
\hline
\end{tabular}
\end{table*}

$\ell^2$ errors and tracer extrema for this test are compared with the horizontal advection results in table~\ref{tab:advection}.  In the terrain following velocity field, tracer accuracy is greatest on the BTF grid.  Using the cubic upwind-biased scheme, errors are about ten times larger on the SLEVE and cut cell grids compared to the BTF grid.

We conclude from this test that accuracy depends upon alignment of the flow with the grid, and accuracy is not significantly reduced by grid distortions.  Error on the BTF grid in the terrain following advection test is comparable with the error on the SLEVE grid in the horizontal advection test.

\subsection{Stratified atmosphere initially at rest}
\label{sec:resting}

An idealised terrain profile is defined along with a stably stratified atmosphere at rest in hydrostatic balance.  The analytic solution is time-invariant, but numerical errors in calculating the horizontal pressure gradient can give rise to spurious velocities which become more severe over steeper terrain \citep{klemp2011}.

The test setup follows the specification by \cite{klemp2011}.  The grid resolution is \(\Delta x = \Delta z^\star = \SI{500}{\meter}\).  All boundary conditions are no normal flow.

The wave-shaped mountain profile has a surface height, $h$, given by
\begin{align}
	h(x) = h_0 \exp \left( - \left( \frac{x}{a} \right)^2 \right) \cos^2 \left( \alpha x \right) \label{eqn:resting:mountain}
\end{align}
where $a = \SI{5}{\kilo\meter}$ is the mountain half-width, $h_0 = \SI{1}{\kilo\meter}$ is the maximum mountain height and $\lambda = \SI{4}{\kilo\meter}$ is the wavelength.  For the optimised SLEVE grid, the large-scale component $h_1$ is specified as
\begin{align}
h_1(x) = \frac{1}{2} h_0 \exp \left( - \left( \frac{x}{a} \right)^2 \right)
\end{align}
and, following \cite{leuenberger2010}, $s_1 = \SI{4}{\kilo\meter}$ is the large scale height, $s_2 = \SI{1}{\kilo\meter}$ is the small scale height, and the optimal exponent value of $n = 1.35$ is used.

The initial thermodynamic conditions are in discrete hydrostatic balance, having a reference potential temperature of $\theta(z = 0) = \SI{288}{\kelvin}$ and constant stability with Brunt-V\"ais\"al\"a frequency $N = \SI{0.01}{\per\second}$ everywhere, except for a more stable layer of $N = \SI{0.02}{\per\second}$ between $\SI{2}{\kilo\meter} \leq z \leq \SI{3}{\kilo\meter}$.  Unlike \citet{klemp2011}, there is no eddy diffusion in the equation set.

The test was integrated forward by 5 hours on the BTF, SLEVE and cut cell grids, and a regular grid with flat terrain.  Maximum vertical velocities are presented in figure~\ref{fig:resting}.  In agreement with \citet{klemp2011}, vertical velocities are larger on more distorted grids.  However, magnitudes are smaller comparing results on the terrain following grids with those from \citet{klemp2011}.  Using the model from \citet{weller-shahrokhi2014}, which includes a curl-free pressure gradient formulation, $w$ reaches a maximum of \SI{0.33}{\meter\per\second} on the BTF grid, compared with a maximum of \(\sim \SI{7}{\meter\per\second}\) found by \citet{klemp2011} using their improved horizontal pressure gradient formulation.

Unlike the result from \citet{klemp2011}, the SLEVE grid does not significantly reduce vertical velocities compared to the BTF grid.  However, errors are two orders of magnitude smaller on the cut cell grid with vertical velocities of \(\sim \SI{1e-4}{\meter\per\second}\).  The smallest error of \(\sim \SI{1e-10}{\meter\per\second}\) is found on the regular grid.

\citet{good2014} found the maximum vertical velocity in their cut cell model was \SI{1e-12}{\meter\per\second}, which is better than any result obtained using the model by \citet{weller-shahrokhi2014}.  It is worth noting that, in the model used by \citet{good2014}, cell centres are in the centre of the uncut cell, resulting in the centre of some cut cells being below the ground (S.-J. Lock 2014, personal communication).  This means that the grid is effectively regular when calculating horizontal and vertical gradients.  This would account for the very small velocities found by \citet{good2014}.

In summary, spurious velocities in the resting atmosphere test were similar on both terrain following grids, with lower errors compared to those from \citet{klemp2011}.  The maximum vertical velocity was significantly decreased on the cut cell grid, so we conclude that non-orthogonality, or lack of alignment of the grid with sufaces of constant gravitational potential are a significant cause of numerical error in this test.

\begin{figure}
	\centering
%	\includegraphics{../fig-resting/fig-resting.pdf}
%
	\caption{Maximum spurious vertical velocity, \(w\) (\si{\meter\per\second}), in the resting atmosphere test with results on BTF, SLEVE, cut cell and regular grids using the model from \citet{weller-shahrokhi2014} which includes a curl-free pressure gradient formulation.}
	\label{fig:resting}
\end{figure}

\subsection{Gravity waves}
The test originally specified by \citet{schaer2002} prescribes flow over terrain with small-scale and large-scale undulations which induces propagating and evanescent gravity waves.

Following \citet{melvin2010}, the domain is \SI{300}{\kilo\meter} wide and \SI{30}{\kilo\meter} high.  The mountain profile has the same form as equation~\ref{eqn:resting:mountain}.  Tests are performed with mountain heights of $h_0 = \SI{250}{\meter}$ and \(h_0 = \SI{500}{\meter}\).  As in the resting atmosphere test, $a = \SI{5}{\kilo\meter}$ is the mountain half-width and $\lambda = \SI{4}{\kilo\meter}$ is the wavelength.  On the optimised SLEVE grid, $s_1 = \SI{5}{\kilo\meter}$ is the large scale height, $s_2 = \SI{2}{\kilo\meter}$ is the small scale height and the optimal exponent value $n = 1.35$ is used.

The initial thermodynamic conditions have a surface temperature of $\theta(z=0) = \SI{288}{\kelvin}$ and constant stability with $N = \SI{0.01}{\per\second}$ everywhere.  A constant horizontal wind $u = \SI{10}{\meter\per\second}$ is prescribed at the inlet boundary.

Sponge layers are added to the upper \SI{10}{\kilo\meter} and leftmost \SI{10}{\kilo\meter} at the inlet boundary to damp the reflection of waves.
The term $\mu \rho \bm{u}$ is subtracted from the momentum (equation \ref{eq:momentum}) where the damping function, \(\mu\), is adapted from \citet{melvin2010} such that
\begin{align}
	\mu(x, z) &= \mu_\mathrm{upper} + \mu_\mathrm{inlet} \\
	\mu_\mathrm{upper}(z) &= \begin{cases}
		\overline{\mu} \sin^2 \left( \frac{\pi}{2} \frac{z - z_B}{H - z_B} \right) & \text{if } z \geq z_B \\
		0 & \text{otherwise} \\
	\end{cases} \\
	\mu_\mathrm{inlet}(x) &= \begin{cases}
		\overline{\mu} \sin^2 \left( \frac{\pi}{2} \frac{x_I - x}{x_I - x_0} \right) & \text{if } x < x_I \\
		0 & \text{otherwise}
	\end{cases}
\end{align}
where $\overline{\mu} = 1.2$ is the damping coefficient, $z_B = \SI{20}{\kilo\meter}$ is the bottom of the sponge layer, $H = \SI{30}{\kilo\meter}$ is the top of the domain, $x_0 = \SI{-150}{\kilo\meter}$ is the leftmost limit of the domain and $x_I = \SI{-140}{\kilo\meter}$ is the rightmost extent of the inlet sponge layer.  The sponge layer is only active on faces whose normal is vertical so that it damps vertical momentum only.

Note that, while the domain itself is \SI{30}{\kilo\meter} in height, for the purposes of generating BTF and SLEVE grids, the domain height is set to \SI{20}{\kilo\meter} because the sponge layer occupies the uppermost \SI{10}{\kilo\meter}.

No normal flow is imposed at the top and bottom boundaries and the outlet has a zero gradient boundary condition.  For the Exner function of pressure, hydrostatic balance is prescribed on all boundaries.  The simulation is integrated forward by 5 hours with a timestep $\Delta t = \SI{8}{\second}$.

Test results are compared between the BTF, SLEVE and cut cell grids.  Vertical velocities on the BTF, SLEVE and cut cell grids are visually indistinguishable (not shown).  They agree with the high resolution mass-conserving semi-implicit semi-Lagrangian solution from \citet{melvin2010}.

The initial thermal profile is subtracted from the potential temperature field at the end of the integration to reveal the structure of thermal anomalies.  Once again, the results are similar on all three grids, and results are shown on the BTF and cut cell grids in figures~\ref{fig:gw-theta}a and \ref{fig:gw-theta}c respectively.  However, examining more closely the anomalies in the lee of the mountain for the cut cell grid, figure~\ref{fig:gw-theta}d shows that the bottommost layer is anomalously warm and the layer above it is anomalously cold.  This feature is not present on the BTF grid (figure~\ref{fig:gw-theta}b) or the SLEVE grid (not shown).  

In a further test, the mountain height is doubled from \SI{250}{\meter} to \SI{500}{\meter} with all other parameter values unchanged.  The same spurious anomaly in potential temperature is again present on the cut cell grid but its amplitude increases.  Figure~\ref{fig:gw-exner-theta} shows vertical profiles of the Exner function of pressure and potential temperature in the lowest \SI{1}{\kilo\meter} in the lee of the mountain at \(x = \SI{50}{\kilo\meter}\).  We find that the decrease in the Exner function of pressure with height is close to linear on all grids.  The potential temperature increases linearly with height on the BTF and SLEVE grids.  On the cut cell grid, the potential temperature anomalies seen in figure~\ref{fig:gw-theta}d appear as a zig--zag in figure~\ref{fig:gw-exner-theta}.

The inaccuracies in potential temperature become invisible when calculating momentum because $\theta$ is interpolated onto cell faces, so discrete hydrostatic balance is maintained.  This is a manifestation of the Lorenz computational mode \citep{arakawa-konor1996,holdaway2013b}.  In models that include moist processes, the Lorenz computational mode can disrupt clouds and generate spurious precipitation \citep{hollingsworth1995}.

To summarize, results of the gravity waves test on all grids are in good agreement with the reference solution from \citet{melvin2010}.  The most prominent errors are found only on the cut cell grid, where the potential temperature errors near the ground excite the Lorenz computational mode.

\begin{figure*}
	\centering
%	\includegraphics{../fig-gravityWaves-theta/fig-gravityWaves-theta.pdf}
%
	\caption{Anomalies in potential temperature in the gravity waves test after 5 hours with a mountain height, \(h_0 = \SI{250}{\meter}\).  The central domain in the lowest \SI{12}{\kilo\meter} is shown on (a) the BTF grid, and (c) the cut cell grid.  The four lowest layers of each grid are shown for (b) BTF, and (d) cut cell grids, using a narrower potential temperature scale.  The results on the SLEVE grid (not shown) are qualitatively identical to results on the BTF grid.}
	\label{fig:gw-theta}
\end{figure*}

\begin{figure}
	\centering
%	\includegraphics{../fig-gravityWaves-sampleLine/fig-gravityWaves-sampleLine.pdf}
%
	\caption{Vertical profiles of the Exner function of pressure, \(\Pi\), and potential temperature, \(\theta\), in the gravity waves test with a mountain height of \(h_0 = \SI{500}{\meter}\).  Exner profile is visually identical on all grids for both mountain heights; for clarity, the Exner profile is only plotted for the BTF grid.  The computational mode is manifested as a zig-zag in potential temperature on the cut cell grid which.   Results on the SLEVE grid (not shown) are qualitatively the same as those on the BTF grid.  The thermal profile with a lesser mountain height of \(h_0 = \SI{250}{\meter}\) (not shown) exhibits a computational mode with smaller amplitude.}
	\label{fig:gw-exner-theta}
\end{figure}

\section{Conclusions}
We have presented a like-for-like comparison between terrain following and cut cell grids using a single model.  Accuracy on the BTF, SLEVE and cut cell grids was evaluated in a series of two-dimensional tests.

Across all tests, a high degree of accuracy was achieved for all grids.  Even on the highly-distorted BTF grid, which have previously been found to give poor results \citep{schaer2002,klemp2011,good2014}, errors were often small in the tests presented here.  In the first two tests, tracers were advected by horizontal and terrain following velocity fields.  We found that the accuracy of the upwind-biased cubic advection scheme depended upon alignment of the flow with the grid rather than on grid distortions.

Spurious vertical velocities were small in the resting atmosphere test, reaching a maximum of $\sim \SI{0.35}{\meter\per\second}$ on the BTF grid, compared to a maximum of $\sim \SI{10}{\meter\per\second}$ found by \citet{klemp2011}.  In the gravity waves test, vertical velocities were in good agreement with the reference solution from \citet{melvin2010} across all grids.

Cut cell grids reduced errors in two of the four tests.  First, in the horizontal advection test, tracer accuracy on the cut cell grid was almost as good as accuracy on a regular grid with no mountain.  Second, in the resting atmosphere test, spurious vertical velocities were two orders of magnitude smaller on the cut cell grid compared with the terrain folllowing grids.

Conversely, in the terrain following advection test, errors were large on the SLEVE and cut cell grids where velocities were misaligned with the grids.  In the gravity waves test, the Lorenz computational mode was manifested as a zig--zag in potential temperature in the lowest layers in the lee of the mountain.  This spurious error was excited only on the cut cell grid.  This test motivates further work to formulate a Charney--Phillips staggering of variables on cut cell grids.

\acknowledgments
I am grateful to my cosupervisors John Methven and Terry Davies for their valuable input, and to Christoph Sch\"{a}r for his assistance in reproducing his advection test results.  I am thankful for the NERC studentship which helped fund this work.  Weller is funded by NERC grant NE/H015698/1.

% REFERENCES

\bibliographystyle{ametsoc2014}
\bibliography{references}

% TABLES


% FIGURES

\end{document}
