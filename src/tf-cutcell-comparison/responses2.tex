\documentclass{article}
\usepackage{fullpage}
\usepackage{url}
\usepackage{natbib}
\usepackage{enumitem}
\usepackage{siunitx}
\usepackage{amsmath}
\usepackage{mathtools}
\usepackage{xcolor}
\newlist{comment}{enumerate}{1}
\setlist[comment,1]{label={\arabic{commenti}.},leftmargin=*,resume=comment}
\newcommand{\TODO}[1]{\textcolor{purple}{TODO: \emph{#1}}}

\begin{document}
\section*{Review \#1}
\begin{quotation}
This revised manuscript is improved from its original version. The Sch\"{a}r mountain wave test with various grid resolutions and the new test for theta advection are quite informative in identifying the potential source of errors in the low-level theta field in the cut-cell mountain-wave simulation. However, as indicated in my previous review, I still believe that this paper should be more focused, with more emphasis on what's new and interesting in these test cases comparing terrain-following and cut-cell coordinates, and less attention to results that are obvious or previously documented. More specific comments and suggestions are included below. 

Sections 4a and 4b – Section 4a presents results for the Sch\"{a}r advection test case for various vertical coordinates and advection schemes; there is little new here that is not already covered in the \citet{schaer2002} paper. The advection tests in section 4b provide an interesting counter example to this test. I would recommend switching the emphasis in these two sections, presenting the advected tracer contours in a new Fig. 3 for the terrain-following advection, and just including results from the Sch\"{a}r case in table 1 for comparison purposes.
\end{quotation}
\begin{quotation}
 Lines 131-147 on cut-cell grid generation - It's still not clear how or why a cell vertex  is moved laterally in constructing the cut-cell mesh. Taking Fig. 2 as an example, it  seems that following the authors' directions to move all vertices lying underneath the  orography on the uniform grid up to the surface would produce four grid cells as  shown in the figure below rather than the three cells shown in Fig. 2. This four-cell structure appears to be consistent with the generation of thin cells as the authors  point out in Fig. 5c. If lateral displacement of a vertex is allowed, how is it decided  whether it should be moved and how is it decided where along the surface it should  be located (i.e. the location of point v in Fig. 2)?
\end{quotation}
We have found an alternative method of generating meshes that does not require the use of the \texttt{snappyHexMesh} utility.  This new method does not move vertices laterally and is more straightforward to describe.  This new method creates meshes that retain the long, thin cells from the old method.  The new method is described in section 2 and figure 2.  All test results on cut cell grids have been updated.

\begin{quotation}
	Section 3b - Finite difference linear advection model - This advection model is solving the same flux form equation (4) as in the previous subsection. Thus, the advection operator in (6) should be in terms of $\partial (u\phi)/\partial x$.  The finite differencing in (6) can also be considered a finite-volume representation, rewriting it as 
	\begin{align*}
		\frac{\partial u \phi}{\partial x} &= \frac{1}{\Delta x} \left( u_{i+1/2} F_{i+1/2} - u_{i - 1/2} F_{i - 1/2}\right), \\
		F_{i+1/2} &= \frac{1}{12} \left( - \phi_{i+2} + 7 \phi_{i+1} + 7 \phi_i - \phi_{i-1} \right).
	\end{align*}
\end{quotation}
Amended as suggested, many thanks.

\begin{quotation}
 Lines 265-267 and table 1 – The authors are unnecessarily complicating their tests  by comparing results from the cut-cell grid and a regular grid. These grids should be identical for this test and merit no further comparative consideration. There is no need to simulate with both grids, show their identical behavior in table 1, and then  state that the same behavior ``is to be expected.''
\end{quotation}
\TODO{agreed, although the comparison with the fourth order scheme in table 1 is perhaps useful?}

\begin{quotation}
 Lines 271-279 and Table 1 – The $\ell_2$ error norms have changed by more than an order  of magnitude from those reported in the original manuscript. What changed?
\end{quotation}
In the first draft, the $\ell_2$ values had not been normalised, such that $\ell_2 = \sqrt{\sum_c \left(\phi - \phi_T\right)^2 \mathcal{V}_c}$.  This was fixed in the second draft so that $\ell_2$ was calculated correctly according to equation (13).
\begin{quotation}
 Lines 280-285 – Reviewer 2 questioned the large change in the $\ell_2$ error norm when 
 the domain width was changed from 301 to 300 km. I share the concern expressed 
 by this reviewer and find the authors' added explanation (``It's likely that changes in 
 the domain width affect the wave power spectrum of the discrete terrain profile.'') to 
 be inadequate. If the results are sensitive to the exact location of the lateral 
 boundaries, then the numerical experiment is not well designed; the boundaries 
 should be moved far enough away from the mountain so that this sensitivity 
 disappears.  Is it possible that the authors shifted the location of the terrain by a half 
 grid interval when they changed the domain width by one grid interval? This would 
 occur if they just defined the center of the terrain profile to be in the center of the 
 domain. If that’s the case, then the results are understandable since numerically, it's a 
 different terrain shape. However, the changing behavior would have nothing to do 
 with the domain width, and thus this sensitivity test should then be removed from 
 the paper altogether. One would get the same sensitivity leaving the domain width 
 fixed and just shifting the terrain location by one-half grid interval.  
\end{quotation}
Yes, the change in domain width causes the terrain to be shifted by $\Delta x / 2$, and it is this which results in a change in accuracy.  We had not intended to perform a sensitivity test, but only to clarify the specification for the \citet{schaer2002} advection test.

The paragraph discussing the sensitivity of CTCS to the change in mountain profile has been removed.  Line \TODO{xxx} clarifies the test setup required to reproduce \citet{schaer2002}'s results.

\begin{quotation}
 Lines 321-322 – When the flow is aligned with the grid there is no cross coordinate 
 flow and therefore the numerics are essentially reduced to a one-dimensional 
 advection equation. I don't see how grid distortion is even an issue. 
\end{quotation}
The advection scheme fits a multidimensional polynomial using a stencil of cell centroids, hence grid distortions could affect the resulting least squares fit.  If the test used a dimensionally-split advection scheme with a terrain-following coordinate transform, then the test would indeed reduce to a one-dimensional problem.  We have provided a more detailed motivation of this test in the introductory paragraph starting at line \TODO{xxx}.

\begin{quotation}
 Lines 364-368 – The authors did not need to run these resting atmosphere 
 simulations to reach the conclusion stated in this paragraph. As stated in my first 
 review, this test case does not challenge the numerics on a cut-cell grid since the horizontal pressure-gradient errors will obviously be reduced in the vicinity of the 
 low level inversion where the coordinates surfaces are perfectly horizontal. A better 
 test case might be to specify an inversion layer that intersects the terrain, since on 
 the cut-cell grid, it's only just adjacent to the terrain where pressure gradient 
 calculations between cell centers are not horizontal.   
\end{quotation}
\TODO{good idea.  zaengl2012 did something similar by raising the mountain to 4km, but we think that simply moving the inversion layer is preferable because the mesh remains constant.  Should we abandon the klemp2011 test entirely or just put less emphasis on its results?}

\begin{quotation}
	Section 4d – Sch\"{a}r mountain-wave tests. These tests are more interesting now that 
 the case has been simulated with a variety of grid resolutions. However, I'm 
 surprised that the analyses of these results seem to focus on behavior far 
 downstream of the mountain, as is evident in Figs. 6 and 7. I'm not aware of any 
 published results that extend beyond x = 25 km in Fig. 6. Since the amplitude of the 
 disturbances are decaying with downstream distance, small inaccuracies in behavior 
 of the absorbing layer and boundary conditions have larger impact such that the 
 relative errors in comparison to the analytic solution become large. (Note, for 
 example the erroneous extension of negative perturbation theta contours to the 
 surface at x = 25 km in Fig. 6.)  All we learn from Fig. 7b is that errors in theta near 
 the surface in the cut-cell cases, which must have been produced in the vicinity of the 
 terrain, continue to be swept downstream. I think it would be more informative to 
 focus on solution differences between the BTF and cut-cell results near the terrain as 
 is done in Fig. 8 for the terrain-following theta advection. The authors could display 
 these difference fields for the various grid resolutions in exactly the same manner as 
 in Fig. 8. The theta advection test presented by the authors is a very nice complement 
 to the gravity-wave tests. It confirms that the BTF solution is behaving properly and 
 therefore lends support to interpreting differences in the two grids as inaccuracies in 
 the cut cell treatment. I would expect that the difference fields in the Sch\"{a}r case are 
 at least qualitatively similar to those exhibited in the theta advection case, which 
 would further confirm the relevance of this advection test as well as the conclusion 
 that errors in the advection of theta through the cut cells may be responsible for the 
 anomalies appearing with this grid.
\end{quotation}
\TODO{the gravity wave amplitude decreases but the advection errors do not.  hence, this is a good choice for the sample line.  we considered plotting heatmaps, but found that it was not as easy to distinguish between the gravity waves and the advection errors.}

\begin{quotation}
 Lines 479-480 – As mentioned above, for the horizontal advection test, shouldn't the  results for the cut-cell and regular grids be identical?
\end{quotation}
This was a mistake: table 1 confirmed that min, max and $\ell_2$ errors were identical on both grids.  This line now reads, `in the horizontal advection test, accuracy was greatest on the cut cell grid'.

\begin{quotation}
 Lines 481-482 – Emphasizing an error reduction of two orders of magnitude for this 
 resting atmosphere case with cut cells compared to a terrain-following coordinate is 
 not very informative. In both cases the errors appear to be quite small, and thus the 
 more relevant question is whether these differences are of any realistic significance. 
\end{quotation}
\TODO{this conclusion may change when we run the new resting atmosphere test}


\section*{Review \#2}
\begin{quotation}
Overall, the authors have been very responsive to my comments and suggestions, and the manuscript improved considerably.
However, there are still a few minor aspects that need attention:
\begin{comment}
\item From equation (7a) it is clear that the symbol mu needs a physical unit which is 1/s. The statement in line 195
that mu is dimensionless is therefore wrong and needs to be corrected.
\end{comment}
\end{quotation}
Removed `dimensionless' on line \TODO{xxx} and now included correct units where $\mu$ is specified on lines \TODO{xxx} and \TODO{xxx}.

\begin{quotation}
	\begin{comment}
	\item Line 241: $\phi$ is defined as the tracer density. Therefore, the definition of $\phi_0$ needs the physical unit \si{\kilogram\per\meter\cubed}
	\end{comment}
\end{quotation}
Added physical units to $\phi_0$ and in the caption for table 1.

\begin{quotation}
	\begin{comment}
	\item Page 18, test c): The initial conditions look incomplete. Please provide information about the initial density or pressure initialization. Do you assume that the pressure at z=0 m is 1000 hPa?
	\end{comment}
\end{quotation}
Yes, amended line \TODO{xxx} to state that $p_0 = \SI{1000}{\hecto\pascal}$, which applies to all tests that use the model.  Line \TODO{xxx} now cites \citet{weller-shahrokhi2014} which explains how the Exner function of pressure is calculated so that it is in discrete hydrostatic balance with the prescribed potential temperature file.

\begin{quotation}
	\begin{comment}
	\item Test d) and e): As for test c), the descriptions of the initial conditions looks incomplete. Is the initial w zero in d,e?
\end{comment}
\end{quotation}
The value of $p_0$ applies to tests (c) and (d).  For test (d), line \TODO{xxx} now states that $w = \SI{0}{\meter\per\second}$.  For test (e), the velocity field is defined by the streamfunction in equation (26), and $w = \SI{0}{\meter\per\second}$ only where $\mathrm{d}h/\mathrm{d}x = 0$.

\begin{quotation}
	\begin{comment}
	\item Line 389: mu needs the physical unit 1/s.
\end{comment}
\end{quotation}
Added, many thanks.

\begin{quotation}
	\begin{comment}
	\item Line 440: It would be better to say 'The functional form of the initial density field ...potential temperature field (Eq. (22)) in the gravity wave test.
	\end{comment}
\end{quotation}
Instead of amending this line, we now refer to the field as a potential temperature field, not a tracer density field.  Further details are provided in response to the subsequent comment.

\begin{quotation}
	\begin{comment}
	\item Line 445: While it is mathematically possible to initialize the tracer density with 288 (needs physical units!) at the ground, it does not make any physical sense. However, it is okay to leave it for this idealized test case, but a more physical choice is clearly more desirable.
	\end{comment}
\end{quotation}
The field is really the potential temperature field from the gravity waves test.  We have modified this section so that the tracer density field is now called a potential temperature field.  Lines \TODO{xxx twice in the abstract, in the intro and conclusion} have also been amended accordingly so that this test is no longer refered to as a tracer advection test..  Equation (\TODO{xxx}) now has $\theta_0 = \SI{288}{\kelvin}$.

\begin{quotation}
	\begin{comment}
	\item Line 449: Do you refer to the normalized $\ell_2$ error here?
	\end{comment}
\end{quotation}
Yes, amended the manuscript to state this.
\begin{quotation}
	\begin{comment}
	\item Check the references for some phrases that need to be capitalized, as e.g. Appalachian in line 571 or Lagrangian in line 567 
	\end{comment}
\end{quotation}
Corrected these, also `Navier-Stokes' and `Met Office Unified Model'.

\bibliographystyle{ametsoc2014}
\bibliography{references}
\end{document}

