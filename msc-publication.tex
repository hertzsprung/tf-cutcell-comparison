\documentclass[twocol]{ametsoc}
\journal{mwr}

\usepackage{xcolor}
\usepackage{amsmath}
\usepackage{mathtools}
\usepackage{siunitx}

\bibpunct{(}{)}{;}{a}{}{,}

\title{Comparison of Terrain Following and Cut Cell Grids with a Non-Hydrostatic Model}
\authors{James Shaw\correspondingauthor{Dept., Institution, Address, City, State/Country.}}
\affiliation{}
\email{j.shaw@pgr.reading.ac.uk}

\abstract{Enter the text of your abstract here.}

\begin{document}
\newcommand{\TODO}[1]{\textcolor{purple}{TODO: \emph{#1}}}

\maketitle

\TODO{Builds on the work of \citet{schaer2002}, \citet{klemp2011}, \citet{weller-shahrokhi2014} (all MWR) \\
Results also compared with \citet{melvin2010} (QRJ), maybe zaengl2012 (MWR), \citet{good2014} (Atmos Sci Letters)}

\section{Introduction}
Representing orography accurately in numerical weather prediction systems is necessary to model downslope winds and local precipitation patterns.  There are two main approaches to represent orography on a grid: terrain following layers and cut cells.  Both methods align cells in vertical columns.

With terrain following (TF) layers the terrain's influence decays with height so that the bottommost layers follow the underlying surface closely while the uppermost layers are flat.  TF layers are usually implemented on a rectangular computational grid using transformed coordinates.

It is well-known that TF coordinates perform badly in the presence of steep orography: as model resolution increases, steeper gradients lead to larger truncation errors in calculating the horizontal pressure gradient which result in spurious winds \citep{dempsey-davis1998,steppeler2002}.

Much work has been done to reduce error associated with TF coordinates: firstly by smoothing the effects of terrain with height \citep{schaer2002,leuenberger-2010,klemp2011} and, secondly, by improving the accuracy in calculating the horizontal pressure gradient itself \citep{klemp2011,zaengl2012}.

Despite their associated numerical errors, TF coordinates are attractive because their rectangular structure is simple to process by computer, boundary layer resolution can be increased with variable spacing of vertical layers \citep{schaer2002}, and cell sizes remain almost constant \citep{jebens2011}.

Cut cells is an alternative method in which cells that lie entirely below the terrain are removed, and those that intersect the surface are modified in shape so that they more closely fit the terrain.  This modification means that some cells become very small, which can reduce computational efficiency \citep{klein2009}, and several approaches have been tried to alleviate the problem \citep{steppeler2002,yamazaki-satomura2010,jebens2011}.

Several studies have found that cut cells produce more accurate results when compared to TF coordinates.  Spurious winds seen in TF coordinates are not present and errors do not increase with steeper terrain \citep{good2014}.  A comparison of TF and cut cells using real initial data by \citet{steppeler2006} found that precipitation patterns, temperature and wind fields were forecast more accurately in the cut cell model.  

Many comparisons of TF and cut cell grids have made between different models.  On the contrary, this study uses the nonhydrostatic model from \citet{weller-shahrokhi2014} to enable a like-for-like comparison between BTF, SLEVE and cut cell grids for idealised, two-dimensional test cases from the literature.  \TODO{final bit of intro: in section 2, section 3 etc etc}

\TODO{TF grids often shown to have lesser accuracy than cut cells in some test cases, often significantly so.  Our results demonstrate that accuracy need not be much reduced and can even be better than cut cells.}

\section{Grids}
\TODO{introduce this section}
\citet{galchen-somerville1975} proposed a basic terrain following (BTF) coordinate defined as 
\begin{equation}
	z = \left( H - h \right) \left( z^\star / H \right) + h
\end{equation}
where, in two dimensions, \(z(x, z^\star)\) is the height of the coordinate surface at level \(z^\star\), \(H\) is the height of the domain, and \(h(x)\) is the height of the terrain surface.  Using this coordinate, the terrain's influence decays linearly with height but disappears only at the top of the domain.

The sigma coordinate transform of \citet{phillips1957} is equivalent to the BTF coordinate transform since they both decay linearly.  However, since they decay with pressure rather than height, sigma coordinates also change with horizontal variations in pressure.

The smooth level vertical (SLEVE) coordinate proposed by \citet{schaer2002} achieves a more regular TF grid in the middle and top of the domain than the BTF coordinate.  The terrain height is split into large-scale and small-scale components, \(h_1\) and \(h_2\), such that \(h = h_1 + h_2\), with each component having a different exponential decay. The transformation is defined as 
\begin{align}
	z &= z^\star + h_1 b_1 + h_2 b_2
\intertext{where the vertical decay functions are given by}
	b_i &= \frac{\sinh \left( \left( H / s_i \right)^n - \left( z^\star / s_i \right)^n \right)}{\sinh \left( H / s_i \right)^n}
\end{align}
with \(s_1\) and \(s_2\) are the scale heights of large-scale and small-scale terrain respectively.  The exponent \(n\) was introduced by \citet{leuenberger2010} in order to increase cell thickness in the layers nearest the ground, allowing longer timesteps and permitting more accurate calculation of parameterised low-level heat and momentum fluxes.  \citet{leuenberger2010} found the exponent has an optimal value of \(n = 1.35\).  Choosing \(n = 1\) gives the decay functions used by \citet{schaer2002}.

\TODO{Explain how SnapCol grid is constructed?}

\section{Numerical method}
\TODO{Hilary suggested adding this as a section but, if I'm not sure what to include.  For now, I've put a summary of the model in the opening para of Results}

\section{Results}
A series of standard, two-dimensional tests were performed over idealised orography.  For each test, results on the BTF, SLEVE and cut cell grid are compared.
We use a finite volume discretisation of the fully-compressible Euler equations described by \citet{weller-shahrokhi2014}.  The model has a curl-free pressure gradient formulation, an upwind-biased cubic advection scheme, and uses a Lorenz staggering of thermodynamic variables.

\subsection{Advection}

Following \citet{schaer2002}, a tracer is transported above wave-shaped terrain by solving the advection equation for a prescribed horizontal wind.  This challenges the accuracy of the advection scheme in the presence of grid distortions.  The spatial domain, terrain profile, wind field, tracer, SLEVE scale heights are given by \citet{schaer2002}.  The optimisation of SLEVE by \citet{leuenberger2010} is not used, so the exponent $n = 1$.  A tracer is placed above and to the left of the mountain with a magnitude of 1 at its centre and 0 at the perimeter.

\TODO{define the continuous wind profile, say that wind is prescribed at cell faces, and explain how we make it discretely non-divergent}

The flux form of the advection equation, \(\partial \phi / \partial t + \nabla \cdot \left( \bm{u} \phi \right) = 0\), is solved using an upwind-biased cubic advection scheme which is non-monotonic and not flux corrected.  The time derivative is solved using a second order Runge-Kutta scheme.  \TODO{I ought to understand this method myself!}

%The tracer is positioned such that it is centred above the mountain at \(t = \SI{5000}{\second}\).

Unlike \citet{schaer2002} who use periodic lateral boundaries, we use a fixed value of 0 at the inlet boundary and zero gradient boundaries elsewhere.
Tests are integrated forward in time for \SI{10000}{\second} with a timestep of \(\Delta t = \SI{25}{\second}\).

The test was executed on the BTF, SLEVE and cut cell grids, and on a regular grid with flat terrain.  Tracer contours at \(t = \SI{0}{\second}, \SI{5000}{\second}\) and \(\SI{10000}{\second}\) are shown in Figure~\ref{fig:advection-tracer}.  The result from \citet{schaer2002} using a fourth-order centred scheme with sigma coordinates (\ref{fig:advection-tracer}a) is compared with the upwind-biased cubic scheme on (\ref{fig:advection-tracer}b) the BTF grid, and (\ref{fig:advection-tracer}c) cut cell grid.

We compare the following grids:
\begin{itemize}
	\item No orography (which tells us how good the model can get)
	\item BTF
	\item SLEVE (only briefly since it's not adding so much to our argument)
	\item SnapCol (briefly because it's the same as noOrography)
\end{itemize}

\begin{figure}
	\centering
	\includegraphics{../msc-publication-fig-advection-tracer/fig-advection-tracer.pdf}
%
	\caption{Horizontally advected tracer contours at \(t = \SI{0}{\second}\), \SI{5000}{\second} and \SI{10000}{\second} using (a) the fourth-order centred difference scheme with sigma coordinates from \citet{schaer2002}, and the upwind-biased cubic scheme on (b) BTF grid, (c) cut cell grid.  Contour intervals are every 0.1.}
	\label{fig:advection-tracer}
\end{figure}

Analysis:
\begin{itemize}
	\item Upwind-biased cubic is qualitatively better than Sch\"ar's 4th order leapfrog on BTF grid (figure~\ref{fig:advection-tracer})
	\item SnapCol error, min and max is almost identical to noOrography (unsurprisingly) (table~\ref{tab:advection})  This agrees with \citet{good2014}
	\item SLEVE error, min and max are part-way between BTF and noOrography (again, unsurprisingly)
	\item noOrography, BTF and SLEVE min and max comparable to \citet{schaer2002}
	\item Conclusion: advection more accurate on more regular grids, but nevertheless satisfactory on all grids
\end{itemize}


\begin{figure}
	\centering
	\includegraphics{../msc-publication-fig-advection-error/fig-advection-error.pdf}
	%
	\caption{Errors in horizontal tracer advection at \(t = \SI{10000}{\second}\) using (a) the fourth-order centred difference scheme with sigma coordinates from \citet{schaer2002}, (b) the upwind-biased cubic scheme on a BTF grid.  Contour intervals are every 0.01 with negative contours denoted by dashed lines.}
	\label{fig:advection-error}
\end{figure}


\subsection{Resting atmosphere}

\begin{itemize}
	\item Motivation: challenge curl-free pressure gradient formulation?
	\item Cite weller-shahrokhi2014 for domain, thermodynamics, discretisation in time and space
\end{itemize}

Analysis:
\begin{itemize}
	\item Spurious \(w\) significantly less than those found by \citet{klemp2011} for BTF: \SI{0.35}{\meter\per\second} vs \SI{10}{\meter\per\second}
	\item SnapCol \(w\) (\SI{1e-3}{\meter\per\second}) significantly than BTF or SLEVE
	\item \TODO{Could also compare with zaengl2012, \citet{good2014}?}
	\item Conclusion: non-orthogonality is a significant cause of numerical error in this test but, again, \TODO{satisfactory?} on all grids
\end{itemize}

\begin{figure*}
	\centering
	\includegraphics{../msc-publication-fig-resting/fig-resting.pdf}
%
	\caption{Maximum spurious vertical velocity \(w\) in the resting atmosphere test with results on (a) BTF, SLEVE, Hybrid Terrain Following (HTF) and Smoothed Terrain Following (STF) coordinates from \citet{klemp2011}, (b) BTF, SLEVE, cut cell and regular grids using the model from \citet{weller-shahrokhi2014}.  Note that vertical scales differ.}
	\label{fig:resting}
\end{figure*}

\subsection{Gravity waves}
\begin{itemize}
	\item \TODO{Motivation: it models a real dynamic process?}
	\item Specify domain, thermodynamics, prescribed inlet wind
	\item Specify sponge layers, BCs
	\item Compare BTF, SLEVE and SnapCol results (SLEVE only briefly as visually identical to BTF)
\end{itemize}

Analysis:
\begin{itemize}
	\item \(w\) contours visually similar on all grids, agree with \citet{melvin2010} (Figure~\ref{fig:gw-w})
	\item \TODO{divergence might help explain computational mode -- it's greater on the SnapCol grid (MSc dissertation Figure 4.15e)}
	\item \(\theta\) anomalies similar on all grids EXCEPT...
	\item ... on SnapCol grid in lee of mountain near the ground. (Figure~\ref{fig:gw-theta})
	\item \TODO{worth mentioning implications of this?  1. reduced stability, although not enough to create vertical motion in this instance 2. would disrupt clouds 2. model has no viscosity so thermal mixing should not occur in theory}
	\item Lorenz computation mode has been excited because Exner sample line at $x = \SI{50}{\kilo\meter}$ is in hydrostatic balance (Figure~\ref{fig:gw-exner-theta})
	\item \TODO{We can speculate on what excites the computational mode, but perhaps better not to?}
	\item No evidence of small cell problem -- \TODO{don't know that we can say much here without another vertical momentum test; our hypothesis about the quasi-horizontal flow isn't backed up with a test case}
	\item Conclusion: results similar on all grids, agree with literature, except for Lorenz computational mode on SnapCol grid
\end{itemize}

\begin{figure}
	\centering
	\includegraphics{../msc-publication-fig-gravityWaves/fig-gravityWaves.pdf}
%
	\caption{Vertical cross section of vertical velocity contours in the gravity waves test after 5 hours on (a) the BTF grid compared with (b) the mass-conserving semi-implicit semi-Lagrangian solution from \citet{melvin2010}.  Contours are every \SI{5e-2}{\meter\per\second} with solid lines denoting ascent and dashed lines descent.}
	\label{fig:gw-w}
\end{figure}

\begin{figure*}
	\centering
	\includegraphics{../msc-publication-fig-gravityWaves-theta/fig-gravityWaves-theta.pdf}
%
	\caption{Anomalies in potential temperature in the gravity waves test after 5 hours.  The central domain in the lowest \SI{12}{\kilo\meter} is shown on (a) the BTF grid, and (c) the cut cell grid.  The four lowest layers of each grid are shown for (b) BTF, and (d) cut cell grids, using a narrower potential temperature scale.}
	\label{fig:gw-theta}
\end{figure*}

\begin{figure}
	\centering
	\includegraphics{../msc-publication-fig-gravityWaves-sampleLine/fig-gravityWaves-sampleLine.pdf}
%
	\caption{Vertical profiles of the Exner function of pressure, \(\Pi\), and potential temperature, \(\theta\), in the gravity waves test.  Exner profile is visually identical on all grids for both mountain heights; for clarity, the Exner profile is only plotted for the BTF grid.  The computational mode is manifested as a zig-zag in potential temperature on the cut cell grid which.  Doubling the mountain height from \SI{250}{\meter} to \SI{500}{\meter} increases the severity of the computational mode but has negligible effect on the BTF grid.  Results on the SLEVE grid (not shown) are qualitatively the same as those on the BTF grid for both mountain heights.}
	\label{fig:gw-exner-theta}
\end{figure}


\section{Conclusions}
\begin{enumerate}
	\item BTF grid isn't as bad as people say it is.  We found that:
	\begin{itemize}
		\item Upwind-biased cubic advection scheme accurately advects a tracer in non-divergent flows (Figure~\ref{fig:advection-tracer}, \ref{fig:advection-error}, Table~\ref{tab:advection})
		\item spurious vertical velocities are small in resting atmosphere (Figure~\ref{fig:resting})
		\item gravity waves results visually as good as reference solution from \citet{melvin2010} (figure~\ref{fig:gw-w})
	\end{itemize}

	\item Cut cell grids can be worse than TF grids:
	\begin{itemize}
		\item Lorenz computational mode found on SnapCol grid only (figure~\ref{fig:gw-theta}, \ref{fig:gw-exner-theta})
	\end{itemize}

	\item Cut cell grids can also be better than TF grids in more artificial test cases:
	\begin{itemize}
		\item SnapCol $w$ two orders of magnitude smaller in resting atmosphere test
		\item SnapCol advection test as good as noOrography
	\end{itemize}
\end{enumerate}

\textbf{With the exception of the Lorenz computational mode on the SnapCol grid in the gravity waves test, results were satisfactory across BTF, SLEVE and SnapCol grids in all three test cases.}

Miscellany:
\begin{itemize}
	\item Advection accuracy depends on alignment of the flow with grid layers (\TODO{we kinda need wobblyTracerAdvection to reach this conclusion})
\end{itemize}

\section{Further work}
\begin{itemize}
	\item Lorenz computational mode motivates formulation of C-P staggering for cut cell grids
	\item Find out what excites the Lorenz computational mode
	\item \TODO{anything else?}
\end{itemize}

\acknowledgments
Start acknowledgments here.

% REFERENCES

\bibliographystyle{ametsoc2014}
\bibliography{references}

% TABLES

\begin{table}[t]
\caption{Min, max and error norms compared with \citet{schaer2002}}
\label{tab:advection}
%
\begin{center}
\begin{tabular}{ l l l l l l }
\hline\hline
& \multicolumn{3}{c}{Cubic upwind-biased} & \multicolumn{2}{c}{Sch\"ar 4th order} \\
& \(\ell^2\) error & min & max & min & max \\
\hline
Analytic  & 0 & 0 & 1 & \multicolumn{1}{c}{---} & \multicolumn{1}{c}{---} \\
BTF 	  & TODO & TODO & TODO & \num{-0.058} & \num{1.001} \\
SLEVE 	  & TODO & TODO & TODO & TODO & TODO \\
SnapCol   & TODO & TODO & TODO & \multicolumn{1}{c}{---} & \multicolumn{1}{c}{---} \\
noOrography & TODO & TODO & TODO & \num{-0.002} & \num{0.984} \\
\hline
\end{tabular}
\end{center}
\end{table}

% FIGURES

\end{document}
